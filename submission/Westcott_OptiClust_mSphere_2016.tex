\documentclass[11pt,]{article}
\usepackage{lmodern}
\usepackage{amssymb,amsmath}
\usepackage{ifxetex,ifluatex}
\usepackage{fixltx2e} % provides \textsubscript
\ifnum 0\ifxetex 1\fi\ifluatex 1\fi=0 % if pdftex
  \usepackage[T1]{fontenc}
  \usepackage[utf8]{inputenc}
\else % if luatex or xelatex
  \ifxetex
    \usepackage{mathspec}
  \else
    \usepackage{fontspec}
  \fi
  \defaultfontfeatures{Ligatures=TeX,Scale=MatchLowercase}
  \newcommand{\euro}{€}
\fi
% use upquote if available, for straight quotes in verbatim environments
\IfFileExists{upquote.sty}{\usepackage{upquote}}{}
% use microtype if available
\IfFileExists{microtype.sty}{%
\usepackage{microtype}
\UseMicrotypeSet[protrusion]{basicmath} % disable protrusion for tt fonts
}{}
\usepackage[margin=1.0in]{geometry}
\usepackage{hyperref}
\PassOptionsToPackage{usenames,dvipsnames}{color} % color is loaded by hyperref
\hypersetup{unicode=true,
            pdftitle={OptiClust: Improved method for assigning amplicon-based sequence data to operational taxonomic units},
            pdfborder={0 0 0},
            breaklinks=true}
\urlstyle{same}  % don't use monospace font for urls
\usepackage{longtable,booktabs}
\usepackage{graphicx,grffile}
\makeatletter
\def\maxwidth{\ifdim\Gin@nat@width>\linewidth\linewidth\else\Gin@nat@width\fi}
\def\maxheight{\ifdim\Gin@nat@height>\textheight\textheight\else\Gin@nat@height\fi}
\makeatother
% Scale images if necessary, so that they will not overflow the page
% margins by default, and it is still possible to overwrite the defaults
% using explicit options in \includegraphics[width, height, ...]{}
\setkeys{Gin}{width=\maxwidth,height=\maxheight,keepaspectratio}
\setlength{\parindent}{0pt}
\setlength{\parskip}{6pt plus 2pt minus 1pt}
\setlength{\emergencystretch}{3em}  % prevent overfull lines
\providecommand{\tightlist}{%
  \setlength{\itemsep}{0pt}\setlength{\parskip}{0pt}}
\setcounter{secnumdepth}{0}

%%% Use protect on footnotes to avoid problems with footnotes in titles
\let\rmarkdownfootnote\footnote%
\def\footnote{\protect\rmarkdownfootnote}

%%% Change title format to be more compact
\usepackage{titling}

% Create subtitle command for use in maketitle
\newcommand{\subtitle}[1]{
  \posttitle{
    \begin{center}\large#1\end{center}
    }
}

\setlength{\droptitle}{-2em}
  \title{\textbf{OptiClust: Improved method for assigning amplicon-based sequence
data to operational taxonomic units}}
  \pretitle{\vspace{\droptitle}\centering\huge}
  \posttitle{\par}
  \author{}
  \preauthor{}\postauthor{}
  \date{}
  \predate{}\postdate{}


% Redefines (sub)paragraphs to behave more like sections
\ifx\paragraph\undefined\else
\let\oldparagraph\paragraph
\renewcommand{\paragraph}[1]{\oldparagraph{#1}\mbox{}}
\fi
\ifx\subparagraph\undefined\else
\let\oldsubparagraph\subparagraph
\renewcommand{\subparagraph}[1]{\oldsubparagraph{#1}\mbox{}}
\fi

\usepackage{helvet} % Helvetica font
\renewcommand*\familydefault{\sfdefault} % Use the sans serif version of the font
\usepackage[T1]{fontenc}

\usepackage[none]{hyphenat}

\usepackage{setspace}
\doublespacing
\setlength{\parskip}{1em}

\usepackage{lineno}

\usepackage{pdfpages}
\usepackage{comment}




\usepackage{multirow}
\usepackage{array}
\newcommand{\bigcell}[2]{\begin{tabular}{@{}#1@{}}#2\end{tabular}}
\usepackage{caption}

\begin{document}
\maketitle

\begin{center}

Running title: OptiClust: Optimized Clustering


\vspace{25mm}
Sarah L. Westcott and Patrick D. Schloss${^\dagger}$

\vspace{30mm}

$\dagger$ To whom correspondence should be addressed: pschloss@umich.edu

Department of Microbiology and Immunology, University of Michigan, Ann Arbor, MI
\end{center}

\newpage

\linenumbers

\subsection{Abstract}\label{abstract}

Assignment of 16S rRNA gene sequences to operational taxonomic units
(OTUs) is a computational bottleneck in the process of analyzing
microbial communities. Although this has been an active area of
research, it has been difficult to overcome the time and memory demands
while improving the quality of the OTU assignments. Here we developed a
new OTU assignment algorithm that iteratively reassigns sequences to new
OTUs to optimize the Matthews correlation coefficient (MCC), a measure
of the quality of OTU assignments. To assess the new algorithm,
OptiClust, we compared it to ten other algorithms using 16S rRNA gene
sequences from two simulated and four natural communities. Using the
OptiClust algorithm, the MCC values averaged 15.2 and 16.5\% higher than
the OTUs generated when we used the average neighbor and distance-based
greedy clustering with VSEARCH, respectively. Furthermore, on average,
OptiClust was 94.6-times faster than the average neighbor algorithm and
just as fast as distance-based greedy clustering with VSEARCH. An
empirical analysis of the efficiency of the algorithms showed that the
time and memory required to perform the algorithm scaled quadratically
with the number of unique sequences in the dataset. The significant
improvement in the quality of the OTU assignments over previously
existing methods will significantly enhance downstream analysis by
limiting the splitting of similar sequences into separate OTUs and
merging of dissimilar sequences into the same OTU. The development of
the OptiClust algorithm represents a significant advance that is likely
to have numerous other applications.

\subsection{Importance}\label{importance}

The analysis of microbial communities from diverse environments using
16S rRNA gene sequencing has expanded our knowledge of the biogeography
of microorganisms. An important step in this analysis is the assignment
of sequences into taxonomic groups based on their similarity to
sequences in a database or based on their similarity to each other,
irrespective of a database. In this study, we present a new algorithm
for the latter approach. The algorithm, OptiClust, seeks to optimize a
metric of assignment quality by shuffling sequences between taxonomic
groups. We found that OptiClust produces more robust assignments and
does so in a rapid and memory efficient manner. This advance will allow
for a more robust analysis of microbial communities and the factors that
shape them.

\newpage

\subsection{Introduction}\label{introduction}

Amplicon-based sequencing has provided incredible insights into Earth's
microbial biodiversity (1, 2). It has become common for studies to
include sequencing millions of 16S rRNA gene sequences across hundreds
of samples (3, 4). This is three to four orders of magnitude greater
sequencing depth than was previously achieved using Sanger sequencing
(5, 6). The increased sequencing depth has revealed novel taxonomic
diversity that is not adequately represented in reference databases (1,
3). However, the advance has forced re-engineering of methods to
overcome the rate and memory limiting steps in computational pipelines
that process raw sequences through the generation of tables containing
the number of sequences in different taxa for each sample (7--10). A
critical component to these pipelines has been the assignment of
amplicon sequences to taxonomic units that are ether defined based on
similarity to a reference or operationally based on the similarity of
the sequences to each other within the dataset (11, 12).

A growing number of algorithms have been developed to cluster sequences
into OTUs. These algorithms can be classified into three general
categories. The first category of algorithms has been termed
closed-reference or phylotyping (13, 14). Sequences are compared to a
reference collection and clustered based on the reference sequences that
they are similar to. This approach is fast; however, the method
struggles when a sequence is similar to multiple reference sequences
that may have different taxonomies and when it is not similar to
sequences in the reference (15). The second category of algorithms has
been called \emph{de novo} because they assign sequences to OTUs without
the use of a reference (14). These include hierarchical algorithms such
as nearest, furthest, and average neighbor (16) and algorithms that
employ heuristics such as abundance or distance-based greedy clustering
as implemented in USEARCH (17) or VSEARCH (18), Sumaclust, OTUCLUST
(19), and Swarm (20). \emph{De novo} methods are agglomerative and tend
to be more computationally intense. It has proven difficult to know
which method generates the best assignments. A third category of
algorithm is open reference clustering, which is a hybrid approach (3,
14). Here sequences are assigned to OTUs using closed-reference
clustering and sequences that are not within a threshold of a reference
sequence are then clustered using a \emph{de novo} approach. This
category blends the strengths and weaknesses of the other method and
adds the complication that closed-reference and \emph{de novo}
clustering use different OTU definitions. These three categories of
algorithms take different approaches to handling large datasets to
minimize the time and memory requirements while attempting to assign
sequences to meaningful OTUs.

Several metrics have emerged for assessing the quality of OTU assignment
algorithms. These have included the time and memory required to run the
algorithm (3, 20--22), agreement between OTU assignments and the
sequences' taxonomy (20, 22--32), sensitivity of an algorithm to
stochastic processes (33), the number of OTUs generated by the algorithm
(23, 34), and the ability to regenerate the assignments made by other
algorithms (3, 35). Unfortunately, these methods fail to directly
quantify the quality of the OTU assignments. An algorithm may complete
with minimal time and memory requirements or generate an idealized
number of OTUs, but the composition of the OTUs could be incorrect.
These metrics also tend to be subjective. For instance, a method may
appear to recapitulate the taxonomy of a synthetic community with known
taxonomic structure, but do a poor job when applied to real communities
with poorly defined taxonomic structure or for sequences that are prone
to misclassification. As an alternative, we developed an approach to
objectively benchmark the clustering quality of OTU assignments (13, 15,
36). This approach counts the number of true positives (TP), true
negatives (TN), false positives (FP), and false negatives (FN) based on
the pairwise distances. Sequence pairs that are within the
user-specified threshold and are clustered together represent TPs and
those in different OTUs are FNs. Those sequence pairs that have a
distance larger than the threshold and are not clustered in the same OTU
are TNs and those in the same OTU are FPs. These values can be
synthesized into a single correlation coefficient, the Matthews
correlation coefficient (MCC), which measures the correlation between
observed and predicted classifications and is robust to cases where
there is an uneven distribution across the confusion matrix (37).
Consistently, the average neighbor algorithm was identified as among the
best or the best algorithm. Other hierarchical algorithms such as
furthest and nearest neighbor, which do not permit the formation of FPs
or FNs, respectively, fared significantly worse. The distance-based
greedy clustering as implemented in VSEARCH has also performed well. The
computational resources required to complete the average neighbor
algorithm can be significant for large datasets and so there is a need
for an algorithm that efficiently produces consistently high quality OTU
assignments.

These benchmarking efforts have assessed the quality of the clusters
after the completion of the algorithm. In the current study we developed
and benchmarked a new \emph{de novo} clustering algorithm that uses real
time calculation of the MCC to direct the progress of the clustering.
The result is the OptiClust algorithm, which produces significantly
better sequence assignments while making efficient use of computational
resources.

\subsection{Results}\label{results}

\textbf{\emph{OptiClust algorithm.}} The OptiClust algorithm uses the
pairs of sequences that are within a desired threshold of each other
(e.g.~0.03), a list of all sequence names in the dataset, and the metric
that should be used to assess clustering quality. A detailed description
of the algorithm is provided for a toy dataset in the Supplementary
Material. Briefly, the algorithm starts by placing each sequence either
within its own OTU or into a single OTU. The algorithm proceeds by
interrogating each sequence and re-calculating the metric for the cases
where the sequence stays in its current OTU, is moved to each of the
other OTUs, or is moved into a new OTU. The location that results in the
best clustering quality indicates whether the sequence should remain in
its current OTU or be moved to a different or new OTU. Each iteration
consists of interrogating every sequence in the dataset. Although
numerous options are available for optimizing the clusters and for
assessing the quality of the clusters within the mothur-based
implementation of the algorithm (e.g.~sensitivity, specificity,
accuracy, F1-score, etc.), the default metric for optimization and
assessment is MCC because it includes all four parameters from the
confusion matrix (Figure S1; Table S1). The algorithm continues until
the optimization metric stabilizes or until it reaches a defined
stopping criteria.

\textbf{\emph{OptiClust-generated OTUs are more robust than those from
other methods.}} To evaluate the OptiClust algorithm and compare its
performance to other algorithms, we utilized six datasets including two
synthetic communities and four previously published large datasets
generated from soil, marine, human, and murine samples (Table 1). When
we seeded the OptiClust algorithm with each sequence in a separate OTU
and ran the algorithm until complete convergence, the MCC values
averaged 15.2 and 16.5\% higher than the OTUs using average neighbor and
distance-based greedy clustering (DGC) with VSEARCH, respectively
(Figure 1; Table S1). The number of OTUs formed by the various methods
was negatively correlated with their MCC value (\(\rho\)=-0.47;
p\textless{}0.001). The OptiClust algorithm was considerably faster than
the hierarchical algorithms and somewhat slower than the heuristic-based
algorithms. Across the six datasets, the OptiClust algorithm was
94.6-times faster than average neighbor and just as fast as DGC with
VSEARCH. The human dataset was a challenge for a number of the
algorithms. OTUCLUST and SumaClust were unable to cluster the human
dataset in less than 50 hours and the average neighbor algorithm
required more than 45 GB of RAM. The USEARCH-based methods were unable
to cluster the human data using the 32-bit free version of the software
that limits the amount of RAM to approximately 3.5 GB. These data
demonstrate that OptiClust generated significantly more robust OTU
assignments than existing methods across a diverse collection of
datasets with performance that was comparable to popular methods.

\textbf{\emph{OptiClust stopping criteria.}} By default, the
mothur-based implementation of the algorithm stops when the optimization
metric changes by less than 0.0001; however, this can be altered by the
user. This implementation also allows the user to stop the algorithm if
a maximum number of iterations is exceeded. By default mothur uses a
maximum value of 100 iterations. The justification for allowing
incomplete convergence was based on the observation that numerous
iterations are performed that extend the time required to complete the
clustering with minimal improvement in clustering (Figure S2). We
evaluated the results of clustering to partial convergence (i.e.~a
change in the MCC value that was less than 0.0001) or until complete
convergence of the MCC value (i.e.~until it did not change between
iterations) when seeding the algorithm with each sequence in a separate
OTU (Figure 1). The small difference in MCC values between the output
from partial and complete convergence resulted in a difference in the
median number of OTUs that ranged between 1.5 and 17.0 OTUs. This
represented a difference of less than 0.15\%. Among the four natural
datasets, between 3 and 6 were needed to achieve partial convergence and
between 8 and 12 iterations were needed to reach full convergence. The
additional steps required between 1.4 and 1.7 times longer to complete
the algorithm. These results suggest that achieving full convergence of
the optimization metric adds computational effort; however, considering
full convergence took between 2 and 17 minutes the extra effort was
relatively small. Although the mothur's default setting is partial
convergence, the remainder of our analysis used complete convergence to
be more conservative.

\textbf{\emph{Effect of seeding OTUs on OptiClust performance.}} By
default the mothur implementation of the OptiClust algorithm starts with
each sequence in a separate OTU. An alternative approach is to start
with all of the sequences in a single OTU. We found that the MCC values
for clusters generated seeding OptiClust with the sequences as a single
OTU were between 0 and 11.5\% lower than when seeding the algorithm with
sequences in separate OTUs (Figure 1). Interestingly, with the exception
of the human dataset (0.2\% more OTUs), the number of OTUs was as much
as 7.0\% lower (mice) than when the algorithm was seeded with sequence
in separate OTUs. Finally, the amount of time required to cluster the
data when the algorithm was seeded with a single OTU was between 1.5 and
2.9-times longer than if sequences were seeded as separate OTUs. This
analysis demonstrates that seeding the algorithm with sequences as
separate OTUs resulted in the best OTU assignments in the shortest
amount of time.

\textbf{\emph{OptiClust-generated OTUs are as stable as those from other
algorithms.}} One concern that many have with \emph{de novo} clustering
algorithms is that their output is sensitive to the initial order of the
sequences because each algorithm must break ties where a sequence could
be assigned to multiple OTUs. An additional concern specific to the
OptiClust algorithm is that it may stabilize at a local optimum. To
evaluate these concerns we compared the results obtained using ten
randomizations of the order that sequences were given to the algorithm.
The median coefficient of variation across the six datasets for MCC
values obtained from the replicate clusterings using OptiClust was 0.1\%
(Figure 1). We also measured the coefficient of variation for the number
of OTUs across the six datasets for each method. The median coefficient
of variation for the number of OTUs generated using OptiClust was 0.1\%.
Confirming our previous results (15), all of the methods we tested were
stable to stochastic processes. Of the methods that involved
randomization, the coefficient of variation for MCC values was
considerably smaller with OptiClust than the other methods and the
coefficient of variation for the number of OTUs was comparable to the
other methods. The variation observed in clustering quality suggested
that the algorithm does not appear to converge to a locally optimum MCC
value. More importantly, the random variation does yield output of a
similarly high quality.

\textbf{\emph{Time and memory required to complete Optimization-based
clustering scales efficiently.}} Although not as important as the
quality of clustering, the amount of time and memory required to assign
sequences to OTUs is a legitimate concern. We observed that the time
required to complete the OptiClust algorithm (Figure 1C) paralleled the
number of pairwise distances that were smaller than 0.03 (Table 1). To
further evaluate how the speed and memory usage scaled with the number
of sequences in the dataset, we measured the time required and maximum
RAM usage to cluster 20, 40, 60, 80, and 100\% of the unique sequences
from each of the natural datasets using the OptiClust algorithm (Figure
2). Within each iteration of the algorithm, each sequence is compared to
every other sequence and each comparison requires a recalculation of the
confusion matrix. This would result in a worst case algorithmic
complexity on the order of N\textsuperscript{3}, where N is the number
of unique sequences. Because the algorithm only needs to keep track of
the sequence pairs that are within the threshold of each other, it is
likely that the implementation of the algorithm is more efficient. To
empirically determine the algorithmic complexity, we fit a power law
function to the data in Figure 2A. We observed power coefficients
between 1.7 and 2.5 for the marine and human datasets, respectively. The
algorithm requires storing a matrix that contains the pairs of sequences
that are close to each other as well as a matrix that indicates which
sequences are clustered together. The memory required to store these
matrices is on the order of N\textsuperscript{2}, where N is the number
of unique sequences. In fact, when we fit a power law function to the
data in Figure 2B, the power coefficients were 1.9. Using the four
natural community datasets, doubling the number of sequences in a
dataset would increase the time required to cluster the data by 4 to
8-fold and increase the RAM required by 4-fold. It is possible that
future improvements to the implementation of the algorithm could improve
this performance.

\textbf{\emph{Cluster splitting heuristic generates OTUs that are as
good as non-split approach.}} We previously described a heuristic to
accelerate OTU assignments where sequences were first classified to
taxonomic groups and within each taxon sequences were assigned to OTUs
using the average neighbor clustering algorithm (13). This method is
similar to open reference clustering except that in our approach all
sequences are subjected to \emph{de novo} clustering following
classification whereas in open reference clustering only those sequences
that cannot be classified are subjected to \emph{de novo} clustering.
Our cluster splitting approach accelerated the clustering and reduced
the memory requirements because the number of unique sequences was
effectively reduced by splitting sequences across taxonomic groups.
Furthermore, because sequences in different taxonomic groups are assumed
to belong to different OTUs they are independent, which permits
parallelization and additional reduction in computation time. Reduction
in clustering quality is encountered in this approach if there are
errors in classification or if two sequences within the desired
threshold belong to different taxonomic groups. It is expected that
these errors would increase as the taxonomic level goes from kingdom to
genus. To characterize the clustering quality, we classified each
sequence at each taxonomic level and calculated the MCC values using
OptiClust, average neighbor, and DGC with VSEARCH when splitting at each
taxonomic level (Figure 3). For each method, the MCC values decreased as
the taxonomic resolution increased; however, the decrease in MCC was not
as large as the difference between clustering methods. As the resolution
of the taxonomic levels increased, the clustering quality remained high,
relative to clusters formed from the entire dataset
(i.e.~kingdom-level). The MCC values when splitting the datasets at the
class and genus levels were within 98.0 and 93.0\%, respectively, of the
MCC values obtained from the entire dataset. These decreases in MCC
value resulted in the formation of as many as 4.7 and 22.5\% more OTUs,
respectively, than were observed from the entire dataset. These errors
were due to the generation of additional false negatives due to
splitting similar sequences into different taxonomic groups. For the
datasets included in the current analysis, the use of the cluster
splitting heuristic was probably not worth the loss in clustering
quality. However, as datasets become larger, it may be necessary to use
the heuristic to clustering the data into OTUs.

\subsection{Discussion}\label{discussion}

Myriad methods have been proposed for assigning 16S rRNA gene sequences
to OTUs. Each claim improved performance based on speed, memory usage,
representation of taxonomic information, and number of OTUs. Each of
these metrics is subjective and do not actually indicate the quality of
the clustering. This led us to propose using the MCC as a metric for
assessing the quality of clustering, post hoc. Here, we described a new
clustering method that seeks to optimize clustering based on an
objective criterion that measures clustering quality in real time. In
the OptiClust algorithm, clustering is driven by optimizing a metric
that assesses whether any two sequences should be grouped into the same
OTU. The result is clusters that are significantly more robust and is
efficient in the time and memory required to cluster the sequences into
OTUs. This makes it more tractable to analyze large datasets without
sacrificing clustering quality as was previously necessary using
heuristic methods.

The cluster optimization procedure is dependent on the metric that is
chosen for optimization. We employed the MCC because it includes the
four values from a confusion matrix. Other algorithms such as the
furthest neighbor and nearest neighbor algorithms minimize the number of
FP and FN, respectively; however, these suffer because the number of FN
and FP are not controlled, respectively (13, 16). Alternatively, one
could optimize based on the sensitivity, specificity, or accuracy, which
are each based on two values from the confusion matrix or they could
optimize based on the F1-score, which is based on three values from the
confusion matrix. Because these metrics do not balance all four
parameters equally, it is likely that one parameter will dominate in the
optimization procedure. For example, optimizing for sensitivity could
lead to a large number of FPs. More FPs increases the number of OTUs
while more FNs collapses OTUs together. It is difficult to know which is
worse since community richness and diversity are linked to the number of
OTUs. In addition, increasing the number of FNs would overstate the
differences between communities while increasing the number of FPs would
overstate their similarity. Therefore, it is important to jointly
minimize the number of FPs and FNs. With this in mind, we decided to
optimize utilizing the MCC. It is possible that other metrics that
balance the four parameters could be developed and employed for
optimization of the clustering.

The OptiClust algorithm is relatively simple. For each sequence it
effectively asks whether the MCC value will increase if the sequence is
moved to a different OTU including creating a new OTU. If the value does
not change, it remains in the current OTU. The algorithm repeats until
the MCC value stabilizes. Assuming that the algorithm is seeded with
each sequence in a separate OTU, it does not appear that the algorithm
converges to a local optimum. Furthermore, execution of the algorithm
with different random number generator seeds produces OTU assignments of
consistently high quality. Future improvements to the implementation of
the algorithm could provide optimization to further improve its speed
and susceptibility to find a local optimum. Users are encouraged to
repeat the OTU assignment several times to confirm that they have found
the best OTU assignments.

Our previous MCC-based analysis of clustering algorithms indicated that
the average neighbor algorithm consistently produced the best OTU
assignments with the DGC-based method using USEARCH also producing
robust OTU assignments. The challenge in using the average neighbor
algorithm is that it requires a large amount of RAM and is
computationally demanding. This led to the development of a splitting
approach that divides the clustering across distinct taxonomic groups
(13). The improved performance provided by the OptiClust algorithm
likely makes such splitting unnecessary for most current datasets. We
have demonstrated that although the OTU assignments made at the genus
level are still better than that of other methods, the quality is not as
good as that found without splitting. The loss of quality is likely due
to misclassification because of limitations in the clustering algorithms
and reference databases. The practical significance of such small
differences in clustering quality remain to be determined; however,
based on the current analysis, it does appear that the number of OTUs is
artificially inflated. Regardless, the best clustering quality should be
pursued given the available computer resources.

The time and memory required to execute the OptiClust algorithm scaled
proportionally to the number of unique sequences raised to the second
power. The power for the time requirement is affected by the similarity
of the sequences in the dataset with datasets containing more similar
sequences having a higher power. Also, the number of unique sequences is
the basis for both the amount of time and memory required to complete
the algorithm. Both the similarity of sequences and number of unique
sequences can be driven by the sequencing error since any errors will
increase the number of unique sequences and these sequences will be
closely related to the perfect sequence. This underscores the importance
of reducing the noise in the sequence data (7). If sequencing errors are
not remediated and are relatively randomly distributed, then it is
likely that the algorithm will require an unnecessary amount of time and
RAM to complete.

The rapid expansion in sequencing capacity has demanded that the
algorithms used to assign 16S rRNA gene sequences to OTUs be efficient
while maintaining robust assignments. Although database-based approaches
have been proposed to facilitate this analysis, they are limited by
their limited coverage of bacterial taxonomy and by the inconsistent
process used to name taxa. The ability to assign sequences to OTUs using
an algorithm that optimizes clustering by directly measuring quality
will significantly enhance downstream analysis. The development of the
OptiClust algorithm represents a significant advance that is likely to
have numerous other applications.

\subsection{Materials and Methods}\label{materials-and-methods}

\textbf{\emph{Sequence data and processing steps.}} To evaluate the
OptiClust and the other algorithms we created two synthetic sequence
collections and four sequence collections generated from previously
published studies. The V4 region of the 16S rRNA gene was used from all
datasets because it is a popular region that can be fully sequenced with
two-fold coverage using the commonly used MiSeq sequencer from Illumina
(7). The method for generating the simulated datasets followed the
approach used by Kopylova et al. (34) and Schloss (36). Briefly, we
randomly selected 10,000 uniques V4 fragments from 16S rRNA gene
sequences that were unique from the SILVA non-redundant database (38). A
community with an even relative abundance profile was generated by
specifying that each sequence had a frequency of 100 reads. A community
with a staggered relative abundance profile was generated by specifying
that the abundance of each sequence was a randomly drawn integer sampled
from a uniform distribution between 1 and 200. Sequence collections
collected from human feces (39), murine feces (40), soil (41), and
seawater (42) were used to characterize the algorithms' performance with
natural communities. These sequence collections were all generated using
paired 150 or 250 nt reads of the V4 region. We re-processed all of the
reads using a common analysis pipeline that included quality score-based
error correction (7), alignment against a SILVA reference database (38,
43), screening for chimeras using UCHIME (9), and classification using a
naive Bayesian classifier with the RDP training set requiring an 80\%
confidence score (10).

\textbf{\emph{Implementation of clustering algorithms.}} In addition to
the OptiClust algorithm we evaluated ten different \emph{de novo}
clustering algorithms. These included three hierarchical algorithms,
average neighbor, nearest neighbor, and furthest neighbor, which are
implemented in mothur (v.1.39.0) (11). Seven heuristic methods were also
used including abundance-based greedy clustering (AGC) and
(distance-based greedy clustering) DGC as implemented in USEARCH (v.6.1)
(17) and VSEARCH (v.2.3.3) ((18){]}, OTUCLUST (v.0.1) (19), SumaClust
(v.1.0.20), and Swarm (v.2.1.9) (20). With the exception of Swarm each
of these methods uses distance-based thresholds to report OTU
assignments. We also evalauted the ability of OptiClust to optimize to
metrics other than MCC. These included accuracy, F1-score, negative
predictive value, positive predictive value, false discovery rate,
senitivity, specificity, the sum of TPs and TNs, the sum of FPs and FNs,
and the number of FNs, FPs, TNs, and TPs (Figure S1; Table S1).

\textbf{\emph{Benchmarking.}} We evaluated the quality of the sequence
clustering, reproducibility of the clustering, the speed of clustering,
and the amount of memory required to complete the clustering. To assess
the quality of the clusters generated by each method, we counted the
cells within a confusion matrix that indicated how well the clusterings
represented the distances between the pair of sequences (13). Pairs of
sequences that were in the same OTU and had a distance less than 3\%
were true positives (TPs), those that were in different OTUs and had a
distance greater than 3\% were true negatives (TNs), those that were in
the same OTU and had a distance greater than 3\% were false positives
(FPs), and those that were in different OTUs and had a distance less
than 3\% were false negatives (FNs). To synthesize the matrix into a
single metric we used the Matthews correlation coefficient using the
\texttt{sens.spec} command in mothur using the following equations.

\[
MCC = \frac{TP \times TN-FP \times FN}{\sqrt{(TP+FP)(TP+FN)(TN+FP)(TN+FN)} }
\]

To assess the reproducibility of the algorithms we randomized the
starting order of each sequence collection ten times and ran each
algorithm on each randomized collection. We then measured the MCC for
each randomization and quantified their percent coefficient of variation
(\% CV; 100 times the ratio of the standard deviation to the mean).

To assess how the the memory and time requirements scaled with the
number of sequences included in each sequence collection, we randomly
subsampled 20, 40, 60, or 80\% of the unique sequences in each
collection. We obtained 10 subsamples at each depth for each dataset and
ran each collection (N= 50 = 5 sequencing depths x 10 replicates)
through each of the algorithms. We used the \texttt{timeout} script to
quantify the maximum RAM used and the amount of time required to process
each sequence collection (\url{https://github.com/pshved/timeout}). We
limited each algorithm to 45 GB of RAM and 50 hours using a single
processor.

\textbf{\emph{Data and code availability.}} The workflow utilized
commands in GNU make (v.3.81), GNU bash (v.4.1.2), mothur (v.1.39.0)
(11), and R (v.3.3.2) (44). Within R we utilized the wesanderson
(v.0.3.2) (45), dplyr (v.0.5.0) (46), tidyr (v.0.6.0) (47), cowplot
(v.0.6.3) (48), and ggplot2 (v.2.2.0.9000) (49) packages. A reproducible
version of this manuscript and analysis is available at
\url{https://github.com/SchlossLab/Westcott_OptiClust_mSphere_2017}.

\subsection{Acknowledgements}\label{acknowledgements}

This work was supported through funding from the National Institutes of
Health to PDS (P30DK034933). SLW designed, implemented, and evaluated
the algorithm. PDS designed and evaluated the algorithm. Both authors
wrote and edited the manuscript.

\newpage

\textbf{Table 1. Description of datasets used to evaluate the OptiClust
algorithm and compare its performance to other algorithms.} Each dataset
contains sequences from the V4 region of the 16S rRNA gene. The number
of distances for each dataset are those that were less than or equal to
0.03. The number of OTUs were determined using the OptiClust algorithm.
The even and staggered datasets were generated by extracting the V4
region from full length reference sequences and the datasets from the
natural communities were generated by sequencing the V4 region using a
Illumina MiSeq with either paired 150 or 250 nt reads.

\begin{longtable}[c]{@{}lcccccc@{}}
\toprule
\textbf{Dataset (Ref.)} & \textbf{Read Length} & \textbf{Samples} &
\textbf{Total Seqs.} & \textbf{Unique Seqs.} & \textbf{Distances} &
\textbf{OTUs}\tabularnewline
\midrule
\endhead
Soil (41) & 150 & 18 & 948,243 & 143,677 & 11,775,167 &
40,216\tabularnewline
Marine (42) & 250 & 7 & 1,384,988 & 75,923 & 12,908,857 &
25,787\tabularnewline
Mice (40) & 250 & 360 & 2,825,495 & 32,447 & 6,988,306 &
2,658\tabularnewline
Human (39) & 250 & 489 & 20,951,841 & 121,281 & 38,544,315 &
11,648\tabularnewline
Even (34, 36) & NA & NA & 1,155,800 & 11,558 & 29,694 &
7,651\tabularnewline
Staggered (34, 36) & NA & NA & 1,156,550 & 11,558 & 29,694 &
7,653\tabularnewline
\bottomrule
\end{longtable}

\newpage

\textbf{Figure 1. Comparison of de novo clustering algorithms.} Plot of
MCC (A), number of OTUs (B), and execution times (C) for the comparison
of \emph{de novo} clustering algorithms when applied to four natural and
two synthetic datasets. The first three columns of each figure contain
the results of clustering the datasets (i) seeding the algorithm with
one sequence per OTU and allowing the algorithm to proceed until the MCC
value no longer changed; (ii) seeding the algorithm with one sequence
per OTU and allowing the algorithm to proceed until the MCC changed by
less than 0.0001; (iii) seeding the algorithm with all of the sequences
in one OTU and allowing the algorithm to proceed until the MCC value no
longer changed. The human dataset could not be clustered by the average
neighbor, Sumaclust, USEARCH, or OTUCLUST with less than 45 GB of RAM or
50 hours of execution time. The median of 10 re-orderings of the data is
presented for each method and dataset. The range of observed values is
indicated by the error bars, which are typically smaller than the
plotting symbol.

\textbf{Figure 2. OptiClust performance} The average execution time (A)
and memory usage (B) required to cluster the four natural datasets. The
confidence intervals indicate the range between the minimum and maximum
values. The y-axis is scaled by the square root to demonstrate the
relationship between the time and memory requirements relative to the
number of unique sequences squared.

\textbf{Figure 3. Effects of taxonomically splitting the datasets on
clustering quality.} The datasets were split at each taxonomic level
based on their classification using a naive Bayesian classifier and
clustered using average neighbor, VSEARCH-based DGC, and OptiClust.

\newpage

\textbf{Table S1. Summary of the average number of true positives, true
negatives, false positives, false negatives and the resulting Matthews
correlation coefficient for each of the clustering methods that were
analyzed in this study for each of the six datasets.} Blank values
indicate that those conditions could not be completed in 50 hours with
45 GB of RAM.

\textbf{Figure S1. The OptiClust algorithm is able to effectively
cluster sequences into OTUs by minimizing or maximizing numerous
metrics.} Plot of MCC (A), number of OTUs (B), and execution times (C)
for the comparison of output from the OptiClust algorithm when to
minimizing or maximizing a variety of parameters when applied to four
natural and two synthetic datasets. Within mothur, OTU assignments can
also be made using other metrics including minimizing false positives
and maximizing the specificity, positive predictive value, and true
negatives; however, these all resulted in sequences being assigned to
separate OTUs, which resulted in no false positives and the maximum
number of true negatives. The error bars indicate the range of values
observed for 10 replicates.

\textbf{Figure S2. The OptiClust algorithm rapidly converges to optimize
the Matthews correlation coefficient.} The six datasets were clustered
into OTUs using the OptiClust algorithm seeking to maximize the Matthews
correlation coefficient. This was repeated 10 times for each dataset.

\textbf{Supplemental text.} Worked example of how OptiClust algorithm
clusters sequences into OTUs.

\newpage

\subsection*{References}\label{references}
\addcontentsline{toc}{subsection}{References}

\hypertarget{refs}{}
\hypertarget{ref-Schloss2016b}{}
1. \textbf{Schloss PD}, \textbf{Girard RA}, \textbf{Martin T},
\textbf{Edwards J}, \textbf{Thrash JC}. 2016. Status of the archaeal and
bacterial census: An update. mBio \textbf{7}:e00201--16.
doi:\href{https://doi.org/10.1128/mbio.00201-16}{10.1128/mbio.00201-16}.

\hypertarget{ref-Locey2016}{}
2. \textbf{Locey KJ}, \textbf{Lennon JT}. 2016. Scaling laws predict
global microbial diversity. Proceedings of the National Academy of
Sciences \textbf{113}:5970--5975.
doi:\href{https://doi.org/10.1073/pnas.1521291113}{10.1073/pnas.1521291113}.

\hypertarget{ref-Rideout2014}{}
3. \textbf{Rideout JR}, \textbf{He Y}, \textbf{Navas-Molina JA},
\textbf{Walters WA}, \textbf{Ursell LK}, \textbf{Gibbons SM},
\textbf{Chase J}, \textbf{McDonald D}, \textbf{Gonzalez A},
\textbf{Robbins-Pianka A}, \textbf{Clemente JC}, \textbf{Gilbert JA},
\textbf{Huse SM}, \textbf{Zhou H-W}, \textbf{Knight R}, \textbf{Caporaso
JG}. 2014. Subsampled open-reference clustering creates consistent,
comprehensive OTU definitions and scales to billions of sequences. PeerJ
\textbf{2}:e545.
doi:\href{https://doi.org/10.7717/peerj.545}{10.7717/peerj.545}.

\hypertarget{ref-Huttenhower2012}{}
4. \textbf{Consortium THMP}. 2012. Structure, function and diversity of
the healthy human microbiome. Nature \textbf{486}:207--214.
doi:\href{https://doi.org/10.1038/nature11234}{10.1038/nature11234}.

\hypertarget{ref-Eckburg2005}{}
5. \textbf{Eckburg PB}, \textbf{Bik EM}, \textbf{Bernstein CN},
\textbf{Purdom E}, \textbf{Dethlefsen L}, \textbf{Sargent M},
\textbf{Gill SR}, \textbf{Nelson KE}, \textbf{Relman DA}. 2005.
Diversity of the human intestinal microbial flora. Science
\textbf{308}:1635--1638.

\hypertarget{ref-Elshahed2008}{}
6. \textbf{Elshahed MS}, \textbf{Youssef NH}, \textbf{Spain AM},
\textbf{Sheik C}, \textbf{Najar FZ}, \textbf{Sukharnikov LO},
\textbf{Roe BA}, \textbf{Davis JP}, \textbf{Schloss PD}, \textbf{Bailey
VL}, \textbf{Krumholz LR}. 2008. Novelty and uniqueness patterns of rare
members of the soil biosphere. Applied and Environmental Microbiology
\textbf{74}:5422--5428.
doi:\href{https://doi.org/10.1128/aem.00410-08}{10.1128/aem.00410-08}.

\hypertarget{ref-Kozich2013}{}
7. \textbf{Kozich JJ}, \textbf{Westcott SL}, \textbf{Baxter NT},
\textbf{Highlander SK}, \textbf{Schloss PD}. 2013. Development of a
dual-index sequencing strategy and curation pipeline for analyzing
amplicon sequence data on the MiSeq Illumina sequencing platform.
Applied and Environmental Microbiology \textbf{79}:5112--5120.
doi:\href{https://doi.org/10.1128/aem.01043-13}{10.1128/aem.01043-13}.

\hypertarget{ref-Schloss2009a}{}
8. \textbf{Schloss PD}. 2009. A high-throughput DNA sequence aligner for
microbial ecology studies. PLOS ONE \textbf{4}:e8230.
doi:\href{https://doi.org/10.1371/journal.pone.0008230}{10.1371/journal.pone.0008230}.

\hypertarget{ref-Edgar2011}{}
9. \textbf{Edgar RC}, \textbf{Haas BJ}, \textbf{Clemente JC},
\textbf{Quince C}, \textbf{Knight R}. 2011. UCHIME improves sensitivity
and speed of chimera detection. Bioinformatics \textbf{27}:2194--2200.
doi:\href{https://doi.org/10.1093/bioinformatics/btr381}{10.1093/bioinformatics/btr381}.

\hypertarget{ref-Wang2007}{}
10. \textbf{Wang Q}, \textbf{Garrity GM}, \textbf{Tiedje JM},
\textbf{Cole JR}. 2007. Naive bayesian classifier for rapid assignment
of rRNA sequences into the new bacterial taxonomy. Applied and
Environmental Microbiology \textbf{73}:5261--5267.
doi:\href{https://doi.org/10.1128/aem.00062-07}{10.1128/aem.00062-07}.

\hypertarget{ref-Schloss2009b}{}
11. \textbf{Schloss PD}, \textbf{Westcott SL}, \textbf{Ryabin T},
\textbf{Hall JR}, \textbf{Hartmann M}, \textbf{Hollister EB},
\textbf{Lesniewski RA}, \textbf{Oakley BB}, \textbf{Parks DH},
\textbf{Robinson CJ}, \textbf{Sahl JW}, \textbf{Stres B},
\textbf{Thallinger GG}, \textbf{Horn DJV}, \textbf{Weber CF}. 2009.
Introducing mothur: Open-source, platform-independent,
community-supported software for describing and comparing microbial
communities. Applied and Environmental Microbiology
\textbf{75}:7537--7541.
doi:\href{https://doi.org/10.1128/aem.01541-09}{10.1128/aem.01541-09}.

\hypertarget{ref-Caporaso2010}{}
12. \textbf{Caporaso JG}, \textbf{Kuczynski J}, \textbf{Stombaugh J},
\textbf{Bittinger K}, \textbf{Bushman FD}, \textbf{Costello EK},
\textbf{Fierer N}, \textbf{Peña AG}, \textbf{Goodrich JK},
\textbf{Gordon JI}, \textbf{Huttley GA}, \textbf{Kelley ST},
\textbf{Knights D}, \textbf{Koenig JE}, \textbf{Ley RE},
\textbf{Lozupone CA}, \textbf{McDonald D}, \textbf{Muegge BD},
\textbf{Pirrung M}, \textbf{Reeder J}, \textbf{Sevinsky JR},
\textbf{Turnbaugh PJ}, \textbf{Walters WA}, \textbf{Widmann J},
\textbf{Yatsunenko T}, \textbf{Zaneveld J}, \textbf{Knight R}. 2010.
QIIME allows analysis of high-throughput community sequencing data.
Nature Methods \textbf{7}:335--336.
doi:\href{https://doi.org/10.1038/nmeth.f.303}{10.1038/nmeth.f.303}.

\hypertarget{ref-Schloss2011}{}
13. \textbf{Schloss PD}, \textbf{Westcott SL}. 2011. Assessing and
improving methods used in operational taxonomic unit-based approaches
for 16S rRNA gene sequence analysis. Applied and Environmental
Microbiology \textbf{77}:3219--3226.
doi:\href{https://doi.org/10.1128/aem.02810-10}{10.1128/aem.02810-10}.

\hypertarget{ref-NavasMolina2013}{}
14. \textbf{Navas-Molina JA}, \textbf{Peralta-Sánchez JM},
\textbf{González A}, \textbf{McMurdie PJ}, \textbf{Vázquez-Baeza Y},
\textbf{Xu Z}, \textbf{Ursell LK}, \textbf{Lauber C}, \textbf{Zhou H},
\textbf{Song SJ}, \textbf{Huntley J}, \textbf{Ackermann GL},
\textbf{Berg-Lyons D}, \textbf{Holmes S}, \textbf{Caporaso JG},
\textbf{Knight R}. 2013. Advancing our understanding of the human
microbiome using QIIME, pp. 371--444. \emph{In} Methods in enzymology.
Elsevier BV.

\hypertarget{ref-Westcott2015}{}
15. \textbf{Westcott SL}, \textbf{Schloss PD}. 2015. De novo clustering
methods outperform reference-based methods for assigning 16S rRNA gene
sequences to operational taxonomic units. PeerJ \textbf{3}:e1487.
doi:\href{https://doi.org/10.7717/peerj.1487}{10.7717/peerj.1487}.

\hypertarget{ref-Schloss2005}{}
16. \textbf{Schloss PD}, \textbf{Handelsman J}. 2005. Introducing DOTUR,
a computer program for defining operational taxonomic units and
estimating species richness. Applied and Environmental microbiology
\textbf{71}:1501--1506.

\hypertarget{ref-Edgar2010}{}
17. \textbf{Edgar RC}. 2010. Search and clustering orders of magnitude
faster than BLAST. Bioinformatics \textbf{26}:2460--2461.
doi:\href{https://doi.org/10.1093/bioinformatics/btq461}{10.1093/bioinformatics/btq461}.

\hypertarget{ref-Rognes2016}{}
18. \textbf{Rognes T}, \textbf{Flouri T}, \textbf{Nichols B},
\textbf{Quince C}, \textbf{Mahé F}. 2016. VSEARCH: A versatile open
source tool for metagenomics. PeerJ \textbf{4}:e2584.
doi:\href{https://doi.org/10.7717/peerj.2584}{10.7717/peerj.2584}.

\hypertarget{ref-Albanese2015}{}
19. \textbf{Albanese D}, \textbf{Fontana P}, \textbf{Filippo CD},
\textbf{Cavalieri D}, \textbf{Donati C}. 2015. MICCA: A complete and
accurate software for taxonomic profiling of metagenomic data.
Scientific Reports \textbf{5}:9743.
doi:\href{https://doi.org/10.1038/srep09743}{10.1038/srep09743}.

\hypertarget{ref-Mah2014}{}
20. \textbf{Mahé F}, \textbf{Rognes T}, \textbf{Quince C},
\textbf{Vargas C de}, \textbf{Dunthorn M}. 2014. Swarm: Robust and fast
clustering method for amplicon-based studies. PeerJ \textbf{2}:e593.
doi:\href{https://doi.org/10.7717/peerj.593}{10.7717/peerj.593}.

\hypertarget{ref-Sun2009}{}
21. \textbf{Sun Y}, \textbf{Cai Y}, \textbf{Liu L}, \textbf{Yu F},
\textbf{Farrell ML}, \textbf{McKendree W}, \textbf{Farmerie W}. 2009.
ESPRIT: Estimating species richness using large collections of 16S rRNA
pyrosequences. Nucleic Acids Research \textbf{37}:e76--e76.
doi:\href{https://doi.org/10.1093/nar/gkp285}{10.1093/nar/gkp285}.

\hypertarget{ref-Cai2011}{}
22. \textbf{Cai Y}, \textbf{Sun Y}. 2011. ESPRIT-tree: Hierarchical
clustering analysis of millions of 16S rRNA pyrosequences in quasilinear
computational time. Nucleic Acids Research \textbf{39}:e95--e95.
doi:\href{https://doi.org/10.1093/nar/gkr349}{10.1093/nar/gkr349}.

\hypertarget{ref-Edgar2013}{}
23. \textbf{Edgar RC}. 2013. UPARSE: Highly accurate OTU sequences from
microbial amplicon reads. Nature Methods \textbf{10}:996--998.
doi:\href{https://doi.org/10.1038/nmeth.2604}{10.1038/nmeth.2604}.

\hypertarget{ref-Mah2015}{}
24. \textbf{Mahé F}, \textbf{Rognes T}, \textbf{Quince C},
\textbf{Vargas C de}, \textbf{Dunthorn M}. 2015. Swarm v2:
Highly-scalable and high-resolution amplicon clustering. PeerJ
\textbf{3}:e1420.
doi:\href{https://doi.org/10.7717/peerj.1420}{10.7717/peerj.1420}.

\hypertarget{ref-Barriuso2011}{}
25. \textbf{Barriuso J}, \textbf{Valverde JR}, \textbf{Mellado RP}.
2011. Estimation of bacterial diversity using next generation sequencing
of 16S rDNA: A comparison of different workflows. BMC Bioinformatics
\textbf{12}:473.
doi:\href{https://doi.org/10.1186/1471-2105-12-473}{10.1186/1471-2105-12-473}.

\hypertarget{ref-Bonder2012}{}
26. \textbf{Bonder MJ}, \textbf{Abeln S}, \textbf{Zaura E},
\textbf{Brandt BW}. 2012. Comparing clustering and pre-processing in
taxonomy analysis. Bioinformatics \textbf{28}:2891--2897.
doi:\href{https://doi.org/10.1093/bioinformatics/bts552}{10.1093/bioinformatics/bts552}.

\hypertarget{ref-Chen2013}{}
27. \textbf{Chen W}, \textbf{Zhang CK}, \textbf{Cheng Y}, \textbf{Zhang
S}, \textbf{Zhao H}. 2013. A comparison of methods for clustering 16S
rRNA sequences into OTUs. PLOS ONE \textbf{8}:e70837.
doi:\href{https://doi.org/10.1371/journal.pone.0070837}{10.1371/journal.pone.0070837}.

\hypertarget{ref-Huse2010}{}
28. \textbf{Huse SM}, \textbf{Welch DM}, \textbf{Morrison HG},
\textbf{Sogin ML}. 2010. Ironing out the wrinkles in the rare biosphere
through improved OTU clustering. Environmental Microbiology
\textbf{12}:1889--1898.
doi:\href{https://doi.org/10.1111/j.1462-2920.2010.02193.x}{10.1111/j.1462-2920.2010.02193.x}.

\hypertarget{ref-May2014}{}
29. \textbf{May A}, \textbf{Abeln S}, \textbf{Crielaard W},
\textbf{Heringa J}, \textbf{Brandt BW}. 2014. Unraveling the outcome of
16S rDNA-based taxonomy analysis through mock data and simulations.
Bioinformatics \textbf{30}:1530--1538.
doi:\href{https://doi.org/10.1093/bioinformatics/btu085}{10.1093/bioinformatics/btu085}.

\hypertarget{ref-Sun2011}{}
30. \textbf{Sun Y}, \textbf{Cai Y}, \textbf{Huse SM}, \textbf{Knight R},
\textbf{Farmerie WG}, \textbf{Wang X}, \textbf{Mai V}. 2011. A
large-scale benchmark study of existing algorithms for
taxonomy-independent microbial community analysis. Briefings in
Bioinformatics \textbf{13}:107--121.
doi:\href{https://doi.org/10.1093/bib/bbr009}{10.1093/bib/bbr009}.

\hypertarget{ref-White2010}{}
31. \textbf{White JR}, \textbf{Navlakha S}, \textbf{Nagarajan N},
\textbf{Ghodsi M-R}, \textbf{Kingsford C}, \textbf{Pop M}. 2010.
Alignment and clustering of phylogenetic markers - implications for
microbial diversity studies. BMC Bioinformatics \textbf{11}:152.
doi:\href{https://doi.org/10.1186/1471-2105-11-152}{10.1186/1471-2105-11-152}.

\hypertarget{ref-AlGhalith2016}{}
32. \textbf{Al-Ghalith GA}, \textbf{Montassier E}, \textbf{Ward HN},
\textbf{Knights D}. 2016. NINJA-OPS: Fast accurate marker gene alignment
using concatenated ribosomes. PLOS Computational Biology
\textbf{12}:e1004658.
doi:\href{https://doi.org/10.1371/journal.pcbi.1004658}{10.1371/journal.pcbi.1004658}.

\hypertarget{ref-He2015}{}
33. \textbf{He Y}, \textbf{Caporaso JG}, \textbf{Jiang X-T},
\textbf{Sheng H-F}, \textbf{Huse SM}, \textbf{Rideout JR}, \textbf{Edgar
RC}, \textbf{Kopylova E}, \textbf{Walters WA}, \textbf{Knight R},
\textbf{Zhou H-W}. 2015. Stability of operational taxonomic units: An
important but neglected property for analyzing microbial diversity.
Microbiome \textbf{3}.
doi:\href{https://doi.org/10.1186/s40168-015-0081-x}{10.1186/s40168-015-0081-x}.

\hypertarget{ref-Kopylova2016}{}
34. \textbf{Kopylova E}, \textbf{Navas-Molina JA}, \textbf{Mercier C},
\textbf{Xu ZZ}, \textbf{Mahé F}, \textbf{He Y}, \textbf{Zhou H-W},
\textbf{Rognes T}, \textbf{Caporaso JG}, \textbf{Knight R}. 2016.
Open-source sequence clustering methods improve the state of the art.
mSystems \textbf{1}:e00003--15.
doi:\href{https://doi.org/10.1128/msystems.00003-15}{10.1128/msystems.00003-15}.

\hypertarget{ref-Schmidt2014}{}
35. \textbf{Schmidt TSB}, \textbf{Rodrigues JFM}, \textbf{Mering C von}.
2014. Limits to robustness and reproducibility in the demarcation of
operational taxonomic units. Environ Microbiol \textbf{17}:1689--1706.
doi:\href{https://doi.org/10.1111/1462-2920.12610}{10.1111/1462-2920.12610}.

\hypertarget{ref-Schloss2016a}{}
36. \textbf{Schloss PD}. 2016. Application of a database-independent
approach to assess the quality of operational taxonomic unit picking
methods. mSystems \textbf{1}:e00027--16.
doi:\href{https://doi.org/10.1128/msystems.00027-16}{10.1128/msystems.00027-16}.

\hypertarget{ref-Matthews1975}{}
37. \textbf{Matthews B}. 1975. Comparison of the predicted and observed
secondary structure of t4 phage lysozyme. Biochimica et Biophysica Acta
(BBA) - Protein Structure \textbf{405}:442--451.
doi:\href{https://doi.org/10.1016/0005-2795(75)90109-9}{10.1016/0005-2795(75)90109-9}.

\hypertarget{ref-Pruesse2007}{}
38. \textbf{Pruesse E}, \textbf{Quast C}, \textbf{Knittel K},
\textbf{Fuchs BM}, \textbf{Ludwig W}, \textbf{Peplies J},
\textbf{Glockner FO}. 2007. SILVA: A comprehensive online resource for
quality checked and aligned ribosomal RNA sequence data compatible with
ARB. Nucleic Acids Research \textbf{35}:7188--7196.
doi:\href{https://doi.org/10.1093/nar/gkm864}{10.1093/nar/gkm864}.

\hypertarget{ref-Baxter2016}{}
39. \textbf{Baxter NT}, \textbf{Ruffin MT}, \textbf{Rogers MAM},
\textbf{Schloss PD}. 2016. Microbiota-based model improves the
sensitivity of fecal immunochemical test for detecting colonic lesions.
Genome Medicine \textbf{8}.
doi:\href{https://doi.org/10.1186/s13073-016-0290-3}{10.1186/s13073-016-0290-3}.

\hypertarget{ref-Schloss2012}{}
40. \textbf{Schloss PD}, \textbf{Schubert AM}, \textbf{Zackular JP},
\textbf{Iverson KD}, \textbf{Young VB}, \textbf{Petrosino JF}. 2012.
Stabilization of the murine gut microbiome following weaning. Gut
Microbes \textbf{3}:383--393.
doi:\href{https://doi.org/10.4161/gmic.21008}{10.4161/gmic.21008}.

\hypertarget{ref-Johnston2016}{}
41. \textbf{Johnston ER}, \textbf{Rodriguez-R LM}, \textbf{Luo C},
\textbf{Yuan MM}, \textbf{Wu L}, \textbf{He Z}, \textbf{Schuur EAG},
\textbf{Luo Y}, \textbf{Tiedje JM}, \textbf{Zhou J},
\textbf{Konstantinidis KT}. 2016. Metagenomics reveals pervasive
bacterial populations and reduced community diversity across the alaska
tundra ecosystem. Front Microbiol \textbf{7}.
doi:\href{https://doi.org/10.3389/fmicb.2016.00579}{10.3389/fmicb.2016.00579}.

\hypertarget{ref-Henson2016}{}
42. \textbf{Henson MW}, \textbf{Pitre DM}, \textbf{Weckhorst JL},
\textbf{Lanclos VC}, \textbf{Webber AT}, \textbf{Thrash JC}. 2016.
Artificial seawater media facilitate cultivating members of the
microbial majority from the gulf of mexico. mSphere
\textbf{1}:e00028--16.
doi:\href{https://doi.org/10.1128/msphere.00028-16}{10.1128/msphere.00028-16}.

\hypertarget{ref-Schloss2010}{}
43. \textbf{Schloss PD}. 2010. The effects of alignment quality,
distance calculation method, sequence filtering, and region on the
analysis of 16S rRNA gene-based studies. PLOS Comput Biol
\textbf{6}:e1000844.
doi:\href{https://doi.org/10.1371/journal.pcbi.1000844}{10.1371/journal.pcbi.1000844}.

\hypertarget{ref-language2015}{}
44. \textbf{R Core Team}. 2015. R: A language and environment for
statistical computing. R Foundation for Statistical Computing, Vienna,
Austria.

\hypertarget{ref-wesanderson}{}
45. \textbf{Ram K}, \textbf{Wickham H}. 2015. wesanderson: A wes
anderson palette generator.

\hypertarget{ref-dplyr}{}
46. \textbf{Wickham H}, \textbf{Francois R}. 2016. dplyr: A grammar of
data manipulation.

\hypertarget{ref-tidyr}{}
47. \textbf{Wickham H}. 2016. tidyr: Easily tidy data with `spread()`
and `gather()` functions.

\hypertarget{ref-cowplot}{}
48. \textbf{Wilke CO}. cowplot: Streamlined plot theme and plot
annotations for 'ggplot2'.

\hypertarget{ref-ggplot2}{}
49. \textbf{Wickham H}. 2009. ggplot2: Elegant graphics for data
analysis. Springer-Verlag New York.

\end{document}
